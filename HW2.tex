\documentclass[]{article}
\usepackage{amsmath}
\usepackage{amsfonts}

%opening
\title{MA3201 HW2}
\author{Thenaesh Elango (A0124772E)}

\begin{document}
\maketitle

\newcommand{\series}{\ensuremath{\sum_{n = 0}^{\infty}}}

\section*{Q1}
	Since $R \times S$ is a Cartesian product of two sets, every element in $R \times S$ can be written as a pair of two elements $(r, s)$, with $r \in R, s \in S$\newline
	Any ideal of $R \times S$, which comprises elements from $R \times S$, may therefore be written as $I \times J$, where $I \subseteq R, J \subseteq S$.\newline
	It remains to show that, for some ideal $I \times J$ of $R \times S$, that $I$ and $J$ are ideals of $R$ and $S$ respectively.\newline\newline
	$\forall (i, j), (i', j') \ I \times S: (i, j) + (i', j') = (i + i', j + j') \in I \times S$, so $\forall i, i' \in I: i + i' \in I$ and $\forall j, j' \in J: j + j' \in J$. $I$ and $J$ are therefore closed under addition.\newline\newline
	$(0, 0) \in I \times J$, so $0 \in I$ and $0 \in J$. $I$ and $J$ therefore have the zero element.\newline\newline
	$\forall (i, j) \in I \times J: -(i, j) = (-i, -j) \in I \times J$, so $\forall i \in I: -i \in I$ and $\forall j \in J: -j \in J$. $I$ and $J$ therefore have additive inverses for each element.\newline\newline
	We have shown that $I$ and $J$ form additive subgroups of $R$ and $S$ respectively. It remains to show that they are closed under $R$- and $S$-multiplication respectively.\newline\newline
	$\forall (r, s) \in R \times S, (i, j) \in I \times J: (r, s)(i, j) = (ri, sj) \in I \times J$, so $\forall r \in R, i \in I: ri \in I$ and $\forall s \in S, j \in J: sj \in J$. Therefore $I$ is closed under $R$-multiplication and $J$ is closed under $S$-multiplication.\newline\newline
	QED.

\section*{Q2}
	We note that $(x)$ is the set of all polynomials $\series a_nx^n$ in $R[x]$ with $a_0 = 0$.\newline\newline
	
	Suppose $R$ is an integral domain.\newline
	Then, for $f(x) = \series a_nx^n \in R[x]$, $g(x) = \series b_xx^n \in R[x]$ where $f(x)g(x) \in (x)$, we have $f(x)g(x) = a_0b_0x^0 + \dots$, so $a_0b_0 = 0$.\newline
	But $R$ is a domain, so $a_0 = 0$ or $b_0 = 0$.\newline
	Therefore $f(x) \in (x)$ or $g(x) \in (x)$, so $(x)$ is a prime ideal.\newline\newline
	
	Suppose $(x)$ is a prime ideal.\newline
	Consider $a,b \in R$ where $ab = 0$.\newline
	Construct $f(x) = \series a_nx^n \in R[x]$ and $g(x) = \series b_nx^n \in R[x]$ where $a_0 = a, b_0 = b$.\newline
	Then $f(x)g(x) = a_0b_0x^0 + \dots = abx^0 + \dots = 0x^0 + \dots \in (x)$.\newline
	But $(x)$ is a prime ideal, so $f(x) \in (x)$ or $g(x) \in (x)$.\newline
	Therefore, $a_0 = 0$ or $b_0 = 0$ i.e. $a = 0$ or $b = 0$, so $R$ is an integral domain.\newline\newline
	
	We have shown that $(x)$ is a prime ideal iff $R$ is an integral domain. QED.\newline\newline
	
	Suppose $R$ is a field. Let $I$ be an ideal such that $(x) \subset I \subseteq R[x]$.\newline
	Then $\exists f(x) = \series a_nx^n \in I-(x)$ where $a_0 \not= 0$.\newline
	Construct $g(x) = \series b_nx^n \in R[x]$ where $b_0 = a_0^{-1}$ (inverse of $a_0$ in $R$) and the remaining $b_n$s are recursively defined as $b_n = -a_0^{-1} \sum_{k=1}^{n} a_kb_{n-k}$.\newline
	Then $f(x)g(x) = \series (\sum_{k=0}^{n} a_kb_{n-k})x^n = 1x^0 + 0x^1 + 0x^2 + \dots = 1 \in R[x]$\newline
	Since $I$ is closed under $R[x]$-multiplication, $1 \in I$, so $I = R[x]$.\newline
	Therefore, $(x)$ is a maximal ideal.\newline\newline
	
	Suppose $(x)$ is a maximal ideal.\newline
	Let $a \in R-\{0\}$ and construct $f(x) = \series a_nx^n \in R[x]-(x)$ where $a_0 = a \not= 0$.\newline
	Then the ideal generated by $f(x)$ and every element of $(x)$ is equal to $R[x]$ and so contains $1$.\newline
	This $1$ must have been obtained by multiplying some element $g(x) \in R[x]$ with $f(x)$, as it is not possible to obtain $1$ by multiplying with elements in $(x)$ which all have their $x^0$ coefficient set to zero.\newline
	Therefore, $\exists g(x) = \series b_nx^n \in R[x]: f(x)g(x) = 1$ i.e. $(\series a_nx^n)(\series b_nx^n) = 1x^0 + 0x^1 + \dots$ i.e. $a_0b_0 = 1$ i.e. $ab_0 = 1$.\newline
	Therefore any element $a \in R-\{0\}$ has an inverse in $R$.\newline
	Therefore, $R$ is a field.\newline\newline
	
	We have shown that $(x)$ is a maximal ideal iff $R$ is a field. QED.

\section*{Q3}
	Suppose neither $I$ nor $J$ is contained in $P$.\newline
	Then $\exists x \in I, y \in J: x, y \not\in P$.\newline
	But $xy \in IJ \subseteq P$. Since $P$ is a prime ideal, $x \in P$ or $y \in P$.\newline
	This is a contradiction.\newline
	Therefore, either $I$ or $J$ is contained in $P$.

\section*{Q4}
	\paragraph{Lemma: There is no unit element in a proper ideal.}
	If there is a unit element $\alpha$ in a proper ideal $I$ of a ring $R$, then $\exists \beta \in R: \alpha\beta = 1 \in I$, since $I$ is closed under $R$-multiplication. Then $I = R$. This contradicts the statement that $I$ is a proper ideal, so there cannot be a unit element in a proper ideal. QED.\newline
	
	Suppose $R$ is a local ring with unique maximal ideal $M$.\newline
	Let $x \in R-M$, and consider the ideal $N$ generated by $M \cup \{x\}$.\newline
	Since $M$ is a maximal ideal, $N = R$, so $1 \in N$.\newline
	From the lemma, we see that no element of $M$ is a unit.\newline
	Since $1 \in N$, it is necessary for some element in $M \cup \{x\}$ to be a unit (possibly $1$ itself), so that $1$ can be generated in $N$.\newline
	Therefore, $x$ must be a unit.\newline\newline
	
	We have shown that every element in $R-M$ is a unit. QED.\newline\newline
	
	Suppose $R$ is a ring in which the set of nonunits forms an ideal $M$.\newline
	Suppose $M \subset I \subseteq R$ for some ideal $I$.\newline
	Then there is an $x \in I$ for which $x$ is a unit in $R$, so $\exists y \in R: xy = 1$.\newline
	But $I$ is closed under $R$-multiplication, so $1 \in I$ i.e. $I = R$.\newline
	Therefore, $M$ is a maximal ideal. It remains to show uniqueness of $M$.\newline\newline
	
	Suppose $N \subset R$ is a maximal ideal. Let $J$ be an ideal such that $N \subseteq J \subset R$.\newline
	By the lemma, $J$ contains no unit element in $R$, so every element in $J$ is a nonunit in $R$.\newline
	But $M$ is the set of all nonunits in $R$, so $J \subseteq M$ i.e. $N \subseteq M$.\newline
	But $N$ is a maximal ideal, so $N = M$.\newline
	Therefore, $M$ is unique.\newline\newline
	
	We have shown $R$ is a local ring with unique maximal ideal $M$. QED.

\section*{Q5}
 \paragraph{a}
	Suppose $a_0$ is a unit in $R$ and $a_1, a_2, \dots$ are nilpotent.\newline
	Then $a_0x^0$ is a unit in $R[x]$ with inverse $b_0x^0$, where $b_0$ is the inverse of $a_0$ in $R$.\newline
	For $i > 0$, assuming $a_i^{n_i} = 0 \in R$, $(a_ix^i)^{n_i} = a_i^{n_i}x^{in_i} = 0x^{in_i} = 0 \in R[x]$, so all the $a_ix^i$s for $i > 0$ are nilpotent.\newline
	We know from HW1 that the sum of a unit and a nilpotent element is a unit.\newline
	Therefore, $a_0 + a_1x$ is nilpotent, and $a_0 + a_1x + \dots a_kx^k $ is nilpotent implies $a_0 + a_1x + \dots + a_kx^k + a_{k+1}x^{k+1}$ is nilpotent.\newline
	By induction on $n$, $a_0 + a_1x + \dots + a_nx^n$ is nilpotent.\newline

\section*{Q6}
	Suppose $P$ is a prime ideal of $R$.\newline
	It follows that $R/P$ is an integral domain.\newline
	It remains to show that $R/P$ is a field, in order to conclude that $P$ is a maximal ideal.\newline\newline
	
	Consider arbitrary nonzero $a + P \in R/P$. We know that $\exists n > 1: a^n + P = a + P$.\newline
	Then $(a^n + P) - (a + P) = 0 + P$ i.e. $(a^n - a) + P = 0 + P$ i.e. $a(a^{n-1} - 1) + P = 0 + P$.\newline
	Since $R/P$ is an integral domain and $a$ is nonzero, we have $(a^{n-1} - 1) + P = 0 + P$ i.e. $a^{n-1} + P = 1 + P$ i.e. $a(a^{n-2}) + P = 1 + P$ i.e. $(a + P)(a^{n-2} + P) = 1 + P$.\newline
	Therefore, every arbitrary nonzero $a + P$ has an inverse in $R/P$, so we can conclude that $R/P$ is a field (since $R/P$ is already a ring by virtue of it being a quotient ring).\newline\newline
	
	Therefore, $P$ is a maximal ideal. QED.
\end{document}
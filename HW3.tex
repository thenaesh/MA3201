\documentclass[]{article}
\usepackage{amsmath}
\usepackage{amsfonts}

%opening
\title{MA3201 HW3}
\author{Thenaesh Elango}

\begin{document}

\newcommand{\overbar}[1]{\mkern 1.5mu\overline{\mkern-1.5mu#1\mkern-1.5mu}\mkern 1.5mu}
\newcommand{\series}{\ensuremath{\sum_{n = 0}^{\infty}}}
\newcommand{\finitesum}{\ensuremath{\sum_{n = 0}^{m}}}
\maketitle

\section*{Q1}
	\paragraph{b}
	Notation: $\overbar{\overbar{p(x)}}$ (there are two bars) denotes the quotient $p(x)/f(x)$ in $R[x]$.\newline Likewise for $q(x)$.\newline\newline
	Let $p(x)$ and $q(x)$ be distinct polynomials in $R[x]$, both of degree less than $n$.\newline
	Then there are $h_1(x), h_2(x) \in R[x]$ such that $p(x) = h_1(x)f(x) + \overbar{\overbar{p(x)}}$ and $q(x) = h_2(x)f(x) + \overbar{\overbar{p(x)}}$.\newline\newline
	Suppose $\overbar{p(x)} = \overbar{q(x)}$.\newline
	Assume without loss of generality that $deg(h_1) \ge deg(h_2)$ and that $deg(p) \ge deg(q)$.\newline
	Then $p(x) - q(x) = h_1(x)f(x) + \overbar{\overbar{p(x)}} - h_2(x)f(x) - \overbar{\overbar{q(x)}} = h_1(x)f(x) - h_2(x)f(x) = (h_1(x) - h_2(x))f(x)$.\newline
	Clearly, $(h_1(x) - h_2(x))$ cannot be the zero polynomial, as otherwise $p(x) = q(x)$, contradicting our assumption that they are distinct.\newline
	Since $f(x)$ is monic, the degree of the product $(h_1(x) - h_2(x))f(x)$ is at least $deg(f) = n$, since the highest degree term cannot be zero.\newline
	Then $p(x) - q(x)$ has degree of at least $n$, so $p(x)$ has degree of at least $n$, contradicting the assumption that $p(n)$ has degree less than $n$.\newline
	Therefore, $\overbar{p(x)} \not= \overbar{q(x)}$. QED.
	
	\paragraph{c}
	Clearly, neither $a(x)$ nor $b(x)$ can be the zero element, as that would make the degree of $f$ no longer $\ge 1$. So both of them are nonzero.\newline
	Also, neither of them can be equal to $f(x)$, as they each need to have a degree $< n$. So $\overbar{a(x)}$ and $\overbar{b(x)}$ are both nonzero elements in the $R[x]/(f(x))$.\newline\newline
	Since $f(x) = a(x)b(x)$, $\overbar{a(x)}\overbar{b(x)} = \overbar{a(x)b(x)} = \overbar{f(x)} = 0 \in R[x]/(f(x))$.\newline
	Since $\overbar{a(x)}$ is an element of $R[x]/(f(x))$ that multiplies with a nonzero element in $R[x]/(f(x))$ to give zero, $\overbar{a(x)}$ is a zero divisor. QED.
	
	\paragraph{d}
	We can write $x^n$ as $(x^n - a) + a = f(x) + a \in R[x]$. Therefore, $\overbar{x^n} = \overbar{a} \in R[x]/(f(x))$.\newline
	Let $k$ be a positive integer such that $a^k = 0 \in R[x]$.\newline
	Then $\overbar{x^{kn}} = \overbar{a^k} = 0 \in R[x]/(f(x))$, so $\overbar{x}$ is nilpotent. QED.


\section*{Q2}
	\paragraph{a}
	The elements may be expressed as:\newline
	$\overbar{p(x)}        = \overbar{-x^2 - 11x + 3}$\newline
	$\overbar{q(x)}        = \overbar{8x^2 - 13x + 5}$\newline
	$\overbar{p(x) + q(x)} = \overbar{7x^2 - 24x + 8}$\newline
	$\overbar{p(x)q(x)}    = \overbar{146x^2 - 236x + 90}$\newline
	
	\paragraph{b}
	We are able to express $x^3 - 2x + 1$ as $(x^2 + x - 1)(x - 1)$.\newline
	Neither $\overbar{x^2 + x - 1}$ nor $\overbar{x - 1}$ is a zero element in $\overbar{E}$, but their product is $\overbar{x^3 - 2x + 1} = 0 \in \overbar{E}$.\newline
	Therefore, there is a nonzero element $\overbar{x - 1} \in \overbar{E}$ that is a zero divisor in $\overbar{E}$, so $\overbar{E}$ is not an integral domain. QED.
	
	\paragraph{c}
	Since $x(-x^2 + 2) = -x^3 + 2x = (-x^3 + 2x - 1) + 1 = (-1)(x^3 - 2x + 1) + 1$, 
	$\overbar{x(-x^2 + 2)} = \overbar{1}$.\newline
	Therefore, we may conclude that $\overbar{x}$ is a unit in $\overbar{E}$.


\section*{Q3}
	\paragraph{}
	We note that $(x)$ is the set of all formal power series $\series a_nx^n$ in $R[[x]]$ with $a_0 = 0$.\newline\newline
	
	Suppose $R$ is an integral domain.\newline
	Then, for $f(x) = \series a_nx^n \in R[[x]]$, $g(x) = \series b_xx^n \in R[[x]]$ where $f(x)g(x) \in (x)$, we have $f(x)g(x) = a_0b_0x^0 + \dots$, so $a_0b_0 = 0$.\newline
	But $R$ is a domain, so $a_0 = 0$ or $b_0 = 0$.\newline
	Therefore $f(x) \in (x)$ or $g(x) \in (x)$, so $(x)$ is a prime ideal.
	
	\paragraph{}
	Suppose $R$ is a field. Let $I$ be an ideal such that $(x) \subset I \subseteq R[[x]]$.\newline
	Then $\exists f(x) = \series a_nx^n \in I-(x)$ where $a_0 \not= 0$.\newline
	Construct $g(x) = \series b_nx^n \in R[[x]]$ where $b_0 = a_0^{-1}$ (inverse of $a_0$ in $R$) and the remaining $b_n$s are recursively defined as $b_n = -a_0^{-1} \sum_{k=1}^{n} a_kb_{n-k}$.\newline
	Then $f(x)g(x) = \series (\sum_{k=0}^{n} a_kb_{n-k})x^n = 1x^0 + 0x^1 + 0x^2 + \dots = 1 \in R[[x]]$\newline
	Since $I$ is closed under $R[[x]]$-multiplication, $1 \in I$, so $I = R[[x]]$.\newline
	Therefore, $(x)$ is a maximal ideal.\newline
	
	\paragraph{}
	Suppose $(x)$ is a maximal ideal.\newline
	Let $a \in R-\{0\}$ and construct $f(x) = \series a_nx^n \in R[[x]]-(x)$ where $a_0 = a \not= 0$.\newline
	Then the ideal generated by $f(x)$ and every element of $(x)$ is equal to $R[[x]]$ and so contains $1$.\newline
	This $1$ must have been obtained by multiplying some element $g(x) \in R[[x]]$ with $f(x)$, as it is not possible to obtain $1$ by multiplying with elements in $(x)$ which all have their $x^0$ coefficient set to zero.\newline
	Therefore, $\exists g(x) = \series b_nx^n \in R[[x]]: f(x)g(x) = 1$ i.e. $(\series a_nx^n)(\series b_nx^n) = 1x^0 + 0x^1 + \dots$ i.e. $a_0b_0 = 1$ i.e. $ab_0 = 1$.\newline
	Therefore any element $a \in R-\{0\}$ has an inverse in $R$.\newline
	Therefore, $R$ is a field.\newline\newline


\section*{Q5}
	\paragraph{}
	Let $F_0$ be the intersection of all the subfields of $F$.
	
	\paragraph{}
	We first show that $F_0$ is a ring.
	\subparagraph{Properties of Addition and Multiplication} Associativity and commutativity of addition and multiplication, as well as the distributive law, follow directly from the definition over $F$ as $F_0 \subseteq F$.
	\subparagraph{Existence of Zero} As $0 \in F$ is in every subfield of $F$, $0 \in F$ is also in $F_0$.
	\subparagraph{Multiplicative Identity} Since $1 \in F_0$ is $1 \in F$, $\forall x \in F_0: 1x = x$.
	\subparagraph{Existence of Negation}For any element $x \in F_0$, its additive inverse $-x$ exists in $F$. But since $x$ is in every subfield of $F$, every subfield $F' \subseteq F$ must contain some additive inverse of $x$, called $y' \in F' \subseteq F$.  We have that $y' = -x$, since $x + (-x) = x + y' = 0 \implies (-x) + x + (-x) = (-x) + x + y' \implies -x = y'$, so this $-x$ exists in every subfield of $F$ i.e. is in $F_0$. Therefore, every element in $F_0$ has an additive inverse in $F_0$.
	\subparagraph{Closure of Addition and Multiplication} Let $x, y \in F_0$. Then for every subfield $F' \subseteq F$, $x, y \in F'$. Since $F'$ is a field, $x+y, xy \in F'$  Since $x+y$ and $xy$ are in every subfield of $F$, they are also in $F_0$. Therefore, $F_0$ is closed under addition and multiplication.
	
	\paragraph{}
	We then show that $F_0$ is a field.\newline
	Since every subfield of $F$ contains $1 \in F$ where $1 \not= 0$, $F_0$ contains $1$ and is not the zero ring.\newline
	Suppose $x$ is a nonzero element in $F$. Then $x$ is in every subfield of $F$. In each of the subfields $F' \subseteq F$, there is an inverse element $y \in F'$ such that $xy' = 1$.\newline
	This inverse element is unique across all the subfields, as if $y, y' \in F$ are both inverses of $x$, then $xy = xy' = 1$ i.e. $x(y - y') = 1 - 1 = 0$ i.e. $y - y\ = 0$ (since $F$ is an integral domain) i.e. $y = y'$.\newline
	Therefore, the inverse $y$ of this nonzero element $x \in F_0$ is in every subfield of $F$ and hence in $F_0$.\newline
	Since every nonzero element in $F_0$ has an inverse, $F_0^\times = F_0 - \{0\}$, so $F_0$ is a field.
	
	\paragraph{}
	Suppose $F_0'$ is another smallest subfield of $F$.\newline
	Then we must have $F_0' \subseteq F_0$.
	Since $F_0$ is the intersection of all the subfields of $F$, $F_0 \subseteq F_0'$.\newline
	Therefore $F_0 = F_0'$, so $F_0$ is indeed the unique smallest subfield of $F$.


\section*{Q6}
	\paragraph{}
	Let $A$ and $B$ denote the field of fractions of $R$ and $F[x,y]$ respectively.\newline
	Then $A = \{\frac{a}{b} : a \in R, b \in R - \{0\} \}$ and $B = \{\frac{a}{b} : a \in F[x,y], b \in F[x,y] - \{0\} \}$.\newline
	
	\paragraph{Lemma}
	We claim that $\forall p, q \in A, \frac{p}{q} \in A$, as long as $q$ is nonzero.\newline
	Suppose $p, q \in A$, with $q$ nonzero.\newline
	Then $\exists \alpha, \beta, \gamma, \delta \in R$ with nonzero $\beta, \delta$ such that $p = \frac{\alpha}{\beta}$ and $q = \frac{\gamma}{\delta}$.\newline
	We may write $\frac{p}{q}$ as $\frac{\alpha}{\beta} \times  \frac{\delta}{\gamma}$, where $\gamma$ is certainly nonzero as $q$ is nonzero.\newline
	Both $\frac{\alpha}{\beta}$ and $\frac{\delta}{\gamma}$ are in $A$, so by multiplicative closure of $A$, $\frac{p}{q} \in A$.\newline
	This proves the claim.
	
	\paragraph{}
	For every $\frac{a}{b} \in A$, $a \in R \subseteq F[x,y], b \in R - \{0\} \subseteq F[x,y] - \{0\}$, so $a \in F[x,y]$.\newline
	We can therefore see the $A \subseteq B$.\newline
	
	\paragraph{}
	Let $\frac{a}{b} \in B$. Then $a \in F[x,y]$ $b \in F[x,y] - \{0\}$.\newline
	We note that $x$ and $y$ are both in $A$. $x$ is in $A$ because $x$ is in $R$, so $\frac{cx}{c} \in A$ for all $c \in R$. $y$ is in $A$ because $x^2, x^2y \in R$, so $\frac{x^2y}{x^2} \in A$.\newline
	We also note that every constant in $F$ may also be expressed as a constant polynomial in $A$, due to the presence of the identity polynomial $\frac{x}{x} \in A$.\newline
	We may thus express every polynomial in $x$ and $y$ as elements in $A$, since $A$ is closed under addition and multiplication.\newline
	Therefore, $a, b \in A$, so $\frac{a}{b} \in A$ by the lemma.\newline
	We can therefore see that $B \subseteq A$.
	
	\paragraph{}
	Therefore, $A = B$ i.e. the field of fractions of $R$ and $F[x,y]$ are the same.
	
	\paragraph{}
	Consider the ideal $I$ generated by all the $x^{n+1}y^n$ for integers $n \ge 0$.\newline
	Each $i \in I$ is a finite sum of polynomials in $x^{n+1}y^n$.\newline
	Let $S$ be a finite set $S \subseteq I$.\newline
	Then $S$ is a finite set of finite polynomials in $x^{n+1}y^n$.\newline
	There is therefore largest degree among all the polynomials in $S$.\newline
	Then $S$ is a finite subset of another ideal $J$ generated by $\{x, x^2y, x^3y^2, \dots, x^{m+1}y^m\}$, where $x^{m+1}y^m$ occurs in the largest degree polynomial in $S$.\newline
	
	\paragraph{}
	Suppose $x^{k+2}y^{k+1} \in (S)$. Then $x^{k+2}y^{k+1} = \sum_{r=0}^{m} s_rx^{r+1}y^r$, where each $s_r$ is a polynomial in $R$.\newline
	But each $s_r$ is given by $x^{a+1}y^a$ for some $a$, so if any term $ s_rx^{r+1}y^r$ has the degree of $y$ equal to $k+1$, then $a + r = k+1$.\newline
	But that means $(a+1) + (r + 1) = k+3$, so the degree of $x$ in any term where the degree of $y$ is $k+1$ must be $k+3$.\newline
	Therefore, $x^{k+2}y^{k+1} \not\in I$.
	
	\paragraph{}
	But $x^{k+2}y^{k+1} \in I$, so $S$ does not generate $I$.\newline
	Therefore, we conclude that no finite subset of $I$ can generate $I$.\newline
	Since $I$ is an ideal of $R$, there is an ideal of $R$ that is not finitely generated.
\end{document}

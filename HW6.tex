\documentclass[]{article}

%opening
\title{MA3201 HW6}
\author{Thenaesh Elango (A0124772E)}

\begin{document}

\maketitle

\section*{Q1}
\subsection*{Lemma 1}
Given that $F$ is a field, we know that $F$ is Noetherian.\newline
By Hilbert's Basis Theorem, we know that $F[x]$ is Noetherian.\newline
By induction on $n$, we also know that $F[x_1, \dots, x_n]$ is Noetherian for finite $n$.\newline
Therefore, any nondecreasing chain of ideals of $F[x_1, \dots, x_n]$ eventually stabilises after finitely many terms.

\subsection*{}
Suppose some ideal $I$ of $F[x_1, \dots, x_n]$ is generated by some possibly infinite $S \subseteq F[x_1, \dots, x_n]$.\newline
Pick elements $x_1, x_2, x_3, \dots$ from $S$ in sequence and let the ideal generated by $x_1, \dots, x_k$ be denoted $I_k$.\newline
Clearly, $I_1 \subseteq I_2 \subseteq \dots$ is a chain of ideals ordered by inclusion.\newline
By Lemma 1, this chain eventually stabilises at some finite $n$ i.e. $I_n = I_{n+1} = \dots$.\newline
Clearly, $I_n \subseteq I$ as it is generated by a subset of the set which generated $I$.\newline
$I$ is therefore a successor of $I_n$ in the chain of ideals.\newline
But since the chain of ideals stabilises after $I_n$, $I_n = I$.\newline
But $I_n$ is generated by a finite subset $x_1,\dots, x_n$ of $S$\newline
Therefore, a finite subset of $S$ is sufficient to generate $I$. QED.


\section*{Q2}
\subsection*{}
$x^2$ is irreducible in $R$ as its only factorisation (except in the trivial cases) in $F[x]$ is $x*x$, which is not permitted as $x \not\in R$.\newline
$x^3$ is irreducible in $R$ as its only factorisations (except in the trivial cases) in $F[x]$ are $x*x^2$, $x^2*x$ and $x*x*x$, all of which are not permitted as $x \not\in R$.

\subsection*{}
Consider $x^6 \in R$.\newline
$x^6 = x^4*x^2$, so $x^6 \in (x^2)$.\newline
But $x^6 = x^3*x^3$, where $x^3 \not\in (x^2)$, so $(x^2)$ is not a prime ideal in $R$.\newline
Therefore, $x^2$ is not a prime element in $R$.\newline
Since there is an irreducible but non-prime element in $R$, $R$ is not a UFD. QED.


\section*{Q3}
\subsection*{}
Suppose $R[x]$ is Noetherian.\newline
Let $I$ be an ideal of $R$.\newline
Let $J$ = $\{ i + xf(x) : i \in I, f \in R[x] \}$.\newline
For any $i + xf(x) \in J$ and $a + xg(x) \i R[x]$:\newline
$(i + xf(x))(a + xg(x)) = ai + af(x) + ig(x) + x^2f(x)g(x) = ai + x(f(x) + g(x) + xf(x)g(x)$.
Since $ai \in I$ by closure of $I$ under $R$-multiplication and $(f(x) + g(x) + xf(x)g(x)) \in R[x]$ by the usual ring properties of $R[x]$, we see that $ai + x(f(x) + g(x) + xf(x)g(x) \in J$ i.e. $(i + xf(x))(a + xg(x)) \in J$ i.e. $J$ is closed under $R[x]$-multiplication.\newline
By similar arguments, it is clear that $J$ is an Abelian group under addition.\newline
Therefore, $J$ is an ideal of $R[x]$, so $J = (g_1, \dots, g_m)$ for some finite $m$.

\subsection*{}


\end{document}
